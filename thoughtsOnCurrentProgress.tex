\documentclass[a4paper]{article}

\usepackage[a4paper]{geometry}

\usepackage{setspace}
\doublespacing

\usepackage[sumlimits,]{amsmath}

\usepackage[]{SIunits}

\begin{document}

%The macros for freestream velocities
\newcommand{\uon}{\unit{0.1}{\metre\per\second}}
\newcommand{\utw}{\unit{0.2}{\metre\per\second}}
\newcommand{\uth}{\unit{0.3}{\metre\per\second}}
\newcommand{\ufo}{\unit{0.4}{\metre\per\second}}
\newcommand{\ufi}{\unit{0.5}{\metre\per\second}}
\newcommand{\usi}{\unit{0.6}{\metre\per\second}}
\newcommand{\use}{\unit{0.7}{\metre\per\second}}
\newcommand{\uei}{\unit{0.8}{\metre\per\second}}
\newcommand{\uni}{\unit{0.9}{\metre\per\second}}
\newcommand{\ute}{\unit{1.0}{\metre\per\second}}
\newcommand{\uel}{\unit{1.1}{\metre\per\second}}
\newcommand{\utv}{\unit{1.2}{\metre\per\second}}
\newcommand{\utt}{\unit{1.3}{\metre\per\second}}

%The macros for plate tilt angle
\newcommand{\ptlt}{$\theta_{plate}$}
\newcommand{\rze}{\unit{0}{\radian}}
\newcommand{\ron}{\unit{\pi/8}{\radian}}
\newcommand{\rtw}{\unit{\pi/4}{\radian}}
\newcommand{\rth}{\unit{3\pi/8}{\radian}}
\newcommand{\rfo}{\unit{\pi/2}{\radian}}

%The macros for commonly used symbols
\newcommand{\ypl}{$y^{+}$} %yPlus
\newcommand{\ured}{$U^{*}$} %Reduced velocity

\section*{Simulations to re-run and why}\label{whyRerun}

Is it possible that the only cases that I need to re-run are the case of tilt angle equals \rtw{} and \rth{} This is possible because the case of \rze{} is already alright and can be easily explained. So too is the case of tilt angle \rfo{} which is basically the simulated version of a setup whose experimental data exists. However, it might be a good idea to re-run all \rth{} cases especially the high velocity cases (\uni{} to \utt). This is because only in the later stages of simulation did I found that the simulation results are different when I ran the same case on OpenFOAM 17.06. Usually I run the simulations using OpenFOAM 2.3.x, but one day, as I re-run the \ute{} case on OpenFOAM 17.06 on my laboratory Lenovo, the simulation produces a large vibration previously absent when ran on OpenFOAM 2.3.x. Then, I found that the case when \uni{} also yields a large and coherent vibration pattern.

Is this just an artifact of continuing from another simulation, but with different boundary conditions? Well, I have attempted to induce stable and coherent vibrations from other freestream velocities as well, even at other tilt angles (for example when \rth), but failed. This suggests that when a stable and coherent vibration appears, it is purely due to the inherent solution of the governing equations by the solver. However, since it is dependent on the solver, using a different solver yields different results. The case when \uni{} for example, has been replicated on the server computer Tornado---even after decomposing the case into 16 parts, and running on a Courant number of about 5, I still manage to obtain similar results to the case ran on 4 cores using OpenFOAM 17.06 on the laboratory Lenovo machine.

The data for tilt angle \ron{} is alright, since we can easily explain the trend found in the data too. Perhaps I only need to re-run the cases where the freestream velocity is equal to \uni{} and \ute{}, since the root-mean-square amplitude of the total lift force is somehow bigger for the case of \ron{} at \uni{} and \ute{}, compared to \rze{}.

But the cases of \rtw{} and \rth{} are hard to explain since the vibration amplitude is similar for both cases past the KVIV lower branch, although the contribution to the total lift that is in-phase with the cylinder displacement is far smaller for the case of \rtw{}.

For the most part, the asymmetry that was present in the cylinder displacement signal when \rtw{} is gone with the revised mesh. The revised mesh changes the distribution of cells in the plate domain of the simulation by deploying a higher concentration of cells nearby the circular cylinder neutral position.

\section*{How to proceed: simulation scheduling}\label{howProceed}
No re-runs are needed for the \rze{} case. I need to re-run the \ron{} case, with focus on freestream velocities \uni{} to \utt{}, using either OpenFOAM 4.1 or OpenFOAM 17.06.

I need to re-run all \rtw{} cases since the asymmetrical displacement signal seems to be an artifact of meshing (for plates with tilt angle larger than \rtw{}, it seems that the mesh needs to have a concentrated band of mesh near the cylinder neutral position, in the plate domain). Preferably, the \rtw{} case should be run on OpenFOAM 2.3.x to elicit a low amplitude response, and possibly a low total lift amplitude too. This is so that when I re-run the \rth{} case, we can obtain a trend where the amplitude response at \rth{} is larger than \rtw{}. If this can take place, it will be easier to explain the trend as follows. The vibration when \rth{} is larger than \rtw{} because the plate tilt is closer to \rfo{}, hence is driven by a much stronger streamwise vortex.

Needless to say, no re-runs are necessary for the case of \rfo{} since the results are already established and compares well to experimental studies by other researchers. Furthermore, I have elicited results for \rfo{} that is most similar to results available in the literature using the new mesh with a circular inflation layer around the cylinder, even at a \ypl{} mean of 5. However, it merits mentioning that this new \rfo{} simulation with a circular inflation layer around the cylinder and detached cylinder-plate domain produces a flow visualisation that is more pixelated compared to its predecessors. Is this an artifact of meshing, or is it the result of overclocking the Courant number to a maximum of close to 5 throughout the simulation, remains a question that merits further investigation, if time permits.

\section*{Why the \rze{} simulation is easy to explain}
Why does the \rze{} simulation results easy to explain, in addition to being a long-awaited result to amplifying the vibration amplitude greater than what was observed when \rfo{}? What differentiates the simulation setup for \rze{} from other tilt angles, in the sense that all other tilt angles except for \rfo{} produces only low vibrations as the reduced velocity \ured{} is increased past the lower branch of the Karman vortex-induced vibration KVIV.

\end{document}
